% ----------------------------------------------------------
% Introdução (exemplo de capítulo sem numeração, mas presente no Sumário)
% ----------------------------------------------------------


\chapter{Introdução}\label{intro}

As mudanças econômicas refletem demandas de mercado. Isso têm impacto fundamental na forma como os recursos são disponibilizados e os serviços são exigidos. Nesse sentido, a tecnologia da informação, e mais especificamente nesse estudo, a entrega de um produto ou serviço deve dar um suporte a altura das cobranças de mercado.

O cenário atual que envolve os processos de desenvolvimento e infraestrutura estão ficando cada vez mais defasados. Tradicionalmente o processo de \textit{deploy}\footnote{Publicação de um determinado software ou serviço para uso.} e gestão de entrega de soluções, tende a tornar a rotina de produção e disponibilização de serviços demasiadamente morosa. Em termos atuais, o usuário final e o próprio mercado demandam agilidade e rápida resposta.

A concepção de um modelo de entrega contínua que permita ganho na produtividade e fácil detecção de erros no processo de \textit{deploy}, tem sido cada vez mais requisitado no meio da gestão da tecnologia de informação aliado a uma cultura \textit{DevOps}\footnote{Development and Operational (Desenvolvimento e Operacional).}. Tomando isso como base, é importante entendermos que os prazos de entrega estão cada vez mais apertados, o que influi no aumento da carga de trabalho, refletindo diretamente no desempenho do profissional, tanto de desenvolvimento quanto de infraestrutura.

O presente projeto tem como foco a formulação de um modelo que permita essa entrega contínua e automatizada no que diz respeito o \textit{deploy} de aplicações em tempo consideravelmente reduzido. Passa-se antes pelo estudo das diversas ferramentas disponíveis no mercado de software, a fim de levantar as possibilidades de aplicação nesse modelo. Sendo assim, serão apresentadas as ferramentas eleitas como mais viáveis para o projeto.

É vital então percebermos a real vantagem no uso do modelo baseado na visão DevOps, a redução na carga de trabalho e a comunicação massiva entre os times de infraestrutura e desenvolvimento dentro de um mesmo projeto, e a entrega ágil do produto trabalhado.


A estrutura desse trabalho está organizado da seguinte forma: primeiramente são traçados os objetivos. Na seção 2 será apresentado o referencial teórico utilizado para embasar o tema e a ideia proposta, de acordo com todas as ferramentas estudadas, a fim de utilização posterior em um cenário de testes. Na seção 3 serão apresentados os métodos utilizados para compor o estudo, assim como as ferramentas utilizadas durante todo o processo. Na seção 4, são apresentados os resultados referente ao projeto. Por fim, menciona-se as referências utilizadas no referido estudo.

\section{Objetivos}
\subsection{Objetivo Geral }
Configurar e construir um cenário de automatização de deploys considerando a Universidade Estadual do Tocantins - UNITINS, como ambiente macro.

\begin{comment}
Este é um comentário
\end{comment}

\subsection{Objetivos Específicos}
\begin{itemize}
	
\item Realizar estudos das principais ferramentas disponíveis;

\item Definir as métricas utilizadas para avaliar o desempenho do cenário proposto;

\item Apresentar as configurações de cada ferramenta adotada.

\end{itemize}

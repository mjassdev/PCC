\chapter{Problema}\label{cap:problema}

A abordagem tradicional torna o processo de deploy lento, estressante e caro, afinal o tempo tem valor. Uma das maiores problemáticas dessa forma de disponibilização é a dificuldade em permitir uma alteração de código e imediatamente depois a sua implementação, isso porque o tempo para realizar essa atualização não permite essa agilidade, a prática adotada geralmente reúne uma série de modificações ou novas funcionalidades e o deploy é feito ao final de um curso, seja semanal, quinzenal ou qualquer que seja a métrica.

Esse problema pode também ser conhecido como \textbf{"A Última Milha"}, ocorre na fase final do processo de desenvolvimento, após a validação dos requisitos e antes da implantação em produção. É exatamente nesse ponto onde são realizados os testes e homologação que o processo é notoriamente dispendioso, imagine gerir dezenas ou centenas de alterações no código onde a cobrança pela disponibilidade delas é imediata. A questão da última milha é mais notória quando adotamos uma visão macro do processo, passamos a perceber quanto tempo de se perde com etapas que podem ser automatizadas, e aliado a isso quanto recurso humano-financeiro poderia estar sendo aplicado em outra área. \cite{sato2014devops}
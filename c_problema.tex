\chapter{Problema}\label{cap:problema}

A abordagem clássica de \textit{deploy}, torna o processo lento, estressante e caro, afinal o tempo tem valor. Uma das maiores problemáticas dessa forma de disponibilização é a dificuldade em permitir uma alteração de código e imediatamente depois a sua implementação, isso porque o tempo para realizar essa atualização não permite essa agilidade, a prática adotada geralmente reúne uma série de modificações ou novas funcionalidades e o deploy a medida que essas alterações vão sendo construídas, seja semanal, quinzenal ou qualquer que seja a métrica.

Esse problema pode também ser conhecido como \textbf{"A Última Milha"}, ocorre na fase final do processo de desenvolvimento, após a validação dos requisitos e antes da implantação em produção. É exatamente nesse ponto onde são realizados os testes e homologação que o processo é notoriamente dispendioso, imagine gerir dezenas ou centenas de alterações no código onde a cobrança pela disponibilidade delas é imediata. A questão da última milha é mais notória quando adotamos uma visão macro do processo, passamos a perceber quanto tempo de se perde com etapas que podem ser automatizadas, e aliado a isso quanto recurso humano-financeiro poderia estar sendo aplicado em outra área. \cite{sato2014devops}

Em se tratando do ambiente da UNITINS, o fato que destaca a necessidade da aplicação desse trabalho, pode ser notado pelo adoção de ambientes que fazem uso de máquinas virtuais (VM). Nessa situação, há a necessidade da instalação de um Sistema Operacional (S.O) sobre uma serviço de virtualização (\textit{hypervisor}) ou disponibilização de acesso externo, via SSH ou FTP. Essa arquitetura diz respeito a cada ambiente individualmente, ou seja, cada aplicação requer que essa configuração seja feita especificamente, assim, evidencia-se o desperdício de tempo reservado para construir cada cenário. Além disso, o uso de recurso computacional é mais elevado se comparado a outras tecnologias mais recentes como \textit{contêineres}, visto que as VMs executam um S.O independente e consequentemente utilizam \textit{kernel} próprio, levando-se em consideração a execução de várias máquinas virtuais no mesmo servidor físico percebemos a necessidade de mais poder de processamento e maior uso de memória.\cite{laureano2006maquinas}
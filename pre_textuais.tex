% ----------------------------------------------------------
% ELEMENTOS PRÉ-TEXTUAIS
% ----------------------------------------------------------
% \pretextual

% ---
% Capa
% ---
\imprimircapa
% ---

% ---
% Folha de rosto
% (o * indica que haverá a ficha bibliográfica)
% ---
\imprimirfolhaderosto
% ---

% ---
% Inserir folha de aprovação
% ---

% Isto é um exemplo de Folha de aprovação, elemento obrigatório da NBR
% 14724/2011 (seção 4.2.1.3). Você pode utilizar este modelo até a aprovação
% do trabalho. Após isso, substitua todo o conteúdo deste arquivo por uma
% imagem da página assinada pela banca com o comando abaixo:
%
% \includepdf{folhadeaprovacao_final.pdf}
%
\begin{folhadeaprovacao}

  		\includegraphics[width=1\textwidth]{imagens/unitins.png}
  		
  		\ABNTEXchapterfont\large   1. INFORMAÇÕES DO ACADÊMICO:
				
  		\normalsize Nome: Matheus José Alves Silva Santos
  		\tab Matrícula: 2015101100100005
  
  		Período VIII	
  		
  		E-mail: matheusetf@gmail.com
  		\tab Telefone: (63) 99285-7729
  		
  		\par
  		\vspace*{0.5cm}     
  		  
  		\ABNTEXchapterfont\large   2. INFORMAÇÕES DO PCC:
		  		\normalsize 
		
		Professor Orientador: Douglas Chagas da Silva
		
		Início das atividades: 13/08/2018 \tab \tab Término das atividades: 20/12/2018
		
		Total de horas semanais dedicadas ao estágio supervisionado: 20 horas
		
		Área de realização do estágio: Infraestrutura Ágil
		
		
		Data: \_\_\_\_$/$\_\_\_\_$/$\_\_\_\_
		\par
		\vspace*{0.5cm}

  		\ABNTEXchapterfont\large   3. PARECER DO PROFESSOR ORIENTADOR:
  		\normalsize 
  		
  		Aprovado $($ \tab $)$ \tab \tab Reprovado $($ \tab $)$  \tab \tab Nota: \_\_\_\_\_
  		
  	 OBSERVAÇÕES:
  		  		
  		  		 DATA: \_\_\_\_$/$\_\_\_\_$/$\_\_\_\_ \tab \assinatura{Professor Orientador}
  		\par
  		\vspace*{0.5cm}
  		 \ABNTEXchapterfont\large   4. PARECER DO COORDENADOR DE ESTÁGIO:
  		   		\normalsize 
  		
  		 OBSERVAÇÕES: 

  		 DATA: \_\_\_\_$/$\_\_\_\_$/$\_\_\_\_ \tab \assinatura{Coord. de Estágio Supervisionado}
  		 
   \begin{center}
    \vspace*{0.1cm}
    {\large\imprimirlocal}
    \par
    {\large\imprimirdata}
    \vspace*{0.5cm}
  \end{center}
  
\end{folhadeaprovacao}
% ---

% ---
% Dedicatória
% ---
\begin{dedicatoria}
   \vspace*{\fill}
   \centering
   \noindent
   \textit{ Este trabalho é dedicado à minha família, pelo apoio incondicional.} \vspace*{\fill}
\end{dedicatoria}
% ---

% ---
% Agradecimentos
% --- Es
\begin{agradecimentos}
À Deus, pela graça de permitir o meu crescimento no campo de estudo em que me sinto feliz e realizado. \\

Aos meus pais e meus irmãos que sempre apostaram em minha capacidade de crescimento e superar as barreiras que a vida impõe.\\

À minha esposa que tem sido compreensiva nos momentos de dificuldades, e companheira nos desafios que surgem.\\

Aos meus amigos e colegas de universidade, que sempre contribuíram para o desenvolvimento do meu conhecimento e pelas parcerias nos projetos e estudos.


\end{agradecimentos}
% ---

% ---
% Epígrafe
% ---
\begin{epigrafe}
    \vspace*{\fill}
	\begin{flushright}
		\textit{``Não só isso, mas também nos gloriamos \\
			nas tribulações, porque sabemos que a tribulação\\
			produz perseverança; a perseverança, um caráter \\
			aprovado; e o caráter aprovado, esperança. \\
			E a esperança não nos decepciona, porque Deus \\
			derramou seu amor em nossos corações, por meio \\
			do Espírito Santo que Ele nos concedeu.``\\
			(Bíblia Sagrada, Romanos 5:3-5)}
			
	\end{flushright}
\end{epigrafe}
% ---

% ---
% RESUMOS
% ---

% resumo em português
\setlength{\absparsep}{18pt} % ajusta o espaçamento dos parágrafos do resumo
\begin{resumo}
A demanda cada vez mais latente pela disponibilização rápida de serviços e aplicações, traz uma alta cobrança sobre a equipes de desenvolvimento e operacional quanto a quase imediata disponibilização de determinado software ao usuário final, aliado a isso a qualidade da aplicação também tem um lugar fundamental no processo. Nesse sentido, um serviço que realiza gestão da construção de uma aplicação, torna a morosidade do processo de deploy tradicional, suprida e reduz os possíveis conflitos entre as equipes de desenvolvimento e operações, a isso conferimos o termo DevOps.

Aliado a isso, o provisionamento da Infraestrutura como código e o uso de contêineres fazem com que todo o ambiente se torne replicável e escalável, assim a agilidade em todo o processo de deploy torne-se evidente e de simples monitoramento.
 

 \textbf{Palavras-chaves}: Automação, DevOps, Modelo Ágil, Integração Contínua.
 
\end{resumo}

% resumo em inglês



% ---
% inserir lista de ilustrações
% ---
\pdfbookmark[0]{\listfigurename}{lof}
\listoffigures*
% ---

% ---
% inserir lista de tabelas
% ---
% \newpage
% \pdfbookmark[0]{\listtablename}{lot}
% \newpage
% \listoftables*
% \cleardoublepage
% ---

% ---
% inserir lista de abreviaturas e siglas
% ---
\begin{siglas}
  \item DevOps - Development and Operational (Desenvolvimento e Operacional).
  \item AWS - Amazon Web Services
  \item SSH - Secure Shell
  \item NETCONF - Network Configuration
  \item CI - Continous Integracion
\end{siglas}
% ---


% ---
% inserir o sumario
% ---
\pdfbookmark[0]{\contentsname}{toc}
\tableofcontents*
\cleardoublepage
% ---


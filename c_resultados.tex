\chapter{Resultados}\label{cap:resultados}

Após análise das ferramentas estudadas e posteriormente eleitas as que melhores se adequaram ao presente estudo, partiu-se para a elaboração e concepção do modelo proposto para entrega de aplicação externa com base em uma visão DevOps.

A figura 5 representa a proposta de modelo de arquitetura para aplicação externa pretendida nesse estudo. O processo que dará início à sequencia de testes e deploy, começará com a requisição do usuário (desenvolvedor) para adquirir um repositório com intuito de armazenar uma determinada aplicação web. Isso quer dizer, que o programador terá acesso a um formulário de requisição (via sistema web) para pretensão de repositório, será retornado ao mesmo o link desse repositório a fim de que o mesmo possa hospedar o seu código. Os códigos e alterações serão enviados ao GitHub. Depois de enviar o código, o Jenkins receberá continuamente os códigos e executará os testes. Sendo os testes realizados com sucesso e tendo retorno positivo, sem erros, o Jenkins realizará o deploy no servidor através do Docker.





















% ---
% Capitulo de METODOLOGIA
% ---




\chapter{Metodologia}\label{cap:metodologia}

No âmbito desse estudo, é proposta uma arquitetura para implantação de hospedagem de aplicações externas. Assim, o levantamento de informações consistiu em avaliar um conjunto de requisitos essenciais para apoiar o modelo apresentado.

Para a eleição das ferramentas escolhidas foi levado em conta, a sua disponibilização de código open source, a flexibilidade no uso e confiabilidade nos resultados, as suas vantagens e desvantagens, a aceitação junto a comunidade científica e a documentação disponível, seja por meio de livros ou artigos publicados ou por contribuições junto a comunidade web.

O processo de escolha das ferramentas teve como ponto inicial estar em consonância com atuabilidade do mercado de TI e ter continua contribuição da comunidade web. Sendo assim, o primeiro critério a ser considerado foi como a ferramenta daria suporte às diversas linguagens de programações disponíveis. Outro fator relevante está atrelado a maneira como o produto pode ser adquirido, mais diretamente, ao custo necessário para fazer uso da ferramenta. A premissa é que as mesmas deveriam possuir versão open source (gratuita) que  dessem suporte às necessidades da arquitetura proposta.

Dessa forma, alguns requisitos foram considerados na escolha, os pontos citados a seguir nortearam o estudo: 

\begin{itemize}
	
	\item Extensibilidade: tomando como fato as rápidas mudanças no cenário tecnológico, a ferramenta deveria poder dar suporte a diferentes linguagens e integrações;
	\item Usabilidade: a ferramenta deveria ser de fácil compreensão e fornecer uma boa experiência ao usuário;
	\item Segurança: é essencial que critérios de segurança sejam inerentes à ferramenta, definição de papéis e usuários.
	
\end{itemize}


\section{Materiais Utilizados}

Tomando por base os critérios mencionados no item 3, passou-se a definição das ferramentas, sendo essas:



\begin{table}[]
	\centering
	\begin{tabular}{l|c}
		\hline
		\multicolumn{1}{c}{DESCRIÇÃO} & ESPECIFICAÇÃO                                                                                            \\ \hline
		Máquina Física                  & \begin{tabular}[c]{@{}c@{}}Mac OS High Sierra 10.13.4 /\\ i5 2.4Ghz 10 GB RAM 1333 MHz DDR3\end{tabular} \\ \hline
		Oracle VirtualBox               & Versão 5.2.8                                                                                             \\ \hline
		Docker                          & Versão                                                                                                   \\ \hline
		Ansible                         & Versão                                                                                                   \\ \hline
		GitHub                          & Serviço Web: www.github.com                                                                              \\ \hline
	\end{tabular}
\caption{Ferramentas utilizadas}
\end{table}

\begin{itemize}
	
\item VirutalBox: Criação de máquinas virtuais para testes. Serão construídas máquinas virtuais para execução dos hosts clientes e servidores;
\item GitHub: Versionamento e repositório. Esse serviço receberá os códigos das aplicações dos desenvolvedores;
\item Ansible: Gerenciamento de configuração e orquestração da rotina. Essa ferramenta será responsável por administrar as rotinas dentro de todo o processo de deploy. Será executado em uma máquina virtual com Sistema Operacional Ubuntu Linux 64 bits.
\item Jenkins: Automatização da execução de tarefas e testes de implantação. O Jenkins será executado em uma máquina virtual com Sistema Operacional Ubuntu Linux 64 bits.
\item Docker: Contêineres para armazenamento das aplicações e testes. O Docker será executado em uma máquina virtual com Sistema Operacional Ubuntu Linux 64 bits.

\end{itemize}


\section{Arquitetura e Pipeline}

Para desenvolvimento do processo de deploy utilizar-se-á a seguinte arquitetura, que envolve desde a referência de repositório Git até o provisionamento e entrega.

\begin{figure} [htb]
	\centering
	\includegraphics[width=1.05\linewidth]{imagens/MODELODEPLOY}
	\caption{Representação da Arquitetura Proposta}
	Fonte: Acervo próprio
	\label{fig:modelodeploy}
\end{figure}


Considerando o modelo apresentado na figura 5, temos como sequenciamento mais detalhado, as etapas a seguir, absorvendo desde o início de uma requisição até a disponibilização da aplicação em servidor de produção.
\begin{enumerate}
	\item No ponto de partida, é pre-requisito a existência de conteúdo codificado em linguagem de programação, que permita o versionamento do projeto.
	\item O desenvolvedor faz a requisição de um novo repositório através de um formulário, que contenha as informações pertinentes, tais como, chave ssh, membros do projeto e nome do mesmo;
	\item A requisição é recebida pelo provedor que verifica os dados de solicitação, avalia a disponibilidade de repositório;
	\item O provedor valida a disponibilidade e cria um novo repositório no GitHub;
	\item O link do repositório é retornado ao usuário juntamente com sua chave de autenticação;
	\item O desenvolvedor envia os commits e o GitHub gera uma notificação de alteração ao Jenkins;
	\item Jenkins recebe a notificação (Git Fetch – Git Merge – Templates Ansible – Dry Run – Commit Check);
	\item O Jenkins realiza teste jundo ao contêiner Docker destinado para tanto;
	\item Jenkins gera um histórico (Notificação aos Administradores);
	\item Administrador analisa os logs de testes;
	\item GitHub gera nova notificação ao Jenkins;
	\item Jenkins gera novo release e gera nova tag de versão;
	\item Estando as alterações aprovadas o Jenkins realiza deploy da aplicação em um contêiner Docker destinado a produção;
\end{enumerate}

Analogamente, um pipeline para essa arquitetura pode ser representado pelo seguinte fluxo:
.

.


O processo inicia com a chamada de uma nova requisição do código-fonte do aplicativo do repositório Github em questão. Após isso faz-se a atualização da versão do projeto levando-se em conta o número de compilação, assim tem-se uma identificação digital específica para essa implantação. Após a atualização da versão do projeto, o Jenkins inicia a construção do código fonte utilizando o Maven.

\section{Métricas}

A definição de mecanismos que forneçam e possibilitem a análise estatística de todo o processo é fundamental dentro deste projeto. Nesse sentido, o uso de métricas é necessário para que se tenha uma visão próxima de cada processo de deploy. Serão adotadas nesse projeto as seguintes métricas expostas na tabela ??:


\begin{table}[htb]
	\begin{tabular}{l|c}
		\hline
		\multicolumn{1}{c}{MÉTRICA} & DESCRIÇÃO \\ \hline
		Tempo de deploy & \begin{tabular}[c]{@{}c@{}}Critério utilizado para avaliar o tempo gasto\\  para uma atualização de aplicação entrar em produção\end{tabular} \\ \hline
		\begin{tabular}[c]{@{}l@{}}Quantidade de linhas de código\\ adicionadas\end{tabular} & \begin{tabular}[c]{@{}c@{}}Critério utilizado para definir o tamanho\\ da atualização enviada.\end{tabular} \\ \hline
		Relação tempo x custo & \begin{tabular}[c]{@{}c@{}}Critério utilizado para mensurar o valor monetário da\\ unidade de tempo gasto pela produtividade\end{tabular} \\ \hline
		\begin{tabular}[c]{@{}l@{}}Desempenho frente a múltiplos\\ deploys\end{tabular} & \begin{tabular}[c]{@{}c@{}}Medição do desempenho obtido frente\\ à um estresse de deploys simultâneos\end{tabular} \\ \hline
		Memória utilizada & \begin{tabular}[c]{@{}c@{}}Quantidade de memória reservada destinada ao\\ processo de deploy\end{tabular} \\ \hline
	\end{tabular}
\caption{Métricas adotadas}
\end{table}

% ---
% Capitulo de METODOLOGIA
% ---




\chapter{Metodologia}\label{cap:metodologia}

No âmbito desse estudo, é proposta uma arquitetura para implantação de hospedagem de aplicações externas. Assim, o levantamento de informações consistiu em avaliar um conjunto de requisitos essenciais para apoiar o modelo apresentado.

Para a eleição das ferramentas escolhidas foi levado em conta, a sua disponibilização de código open source, a flexibilidade no uso e confiabilidade nos resultados, as suas vantagens e desvantagens, a aceitação junto a comunidade científica e a documentação disponível, seja por meio de livros ou artigos publicados ou por contribuições junto a comunidade web.

O processo de escolha das ferramentas teve como ponto inicial estar em consonância com atuabilidade do mercado de TI e ter continua contribuição da comunidade web. Sendo assim, o primeiro critério a ser considerado foi como a ferramenta daria suporte às diversas linguagens de programações disponíveis. Outro fator relevante está atrelado a maneira como o produto pode ser adquirido, mais diretamente, ao custo necessário para fazer uso da ferramenta. A premissa é que as mesmas deveriam possuir versão open source (gratuita) que  dessem suporte às necessidades da arquitetura proposta.

Dessa forma, alguns requisitos foram considerados na escolha, os pontos citados a seguir nortearam o estudo: 

\begin{itemize}
	
	\item Extensibilidade: tomando como fato as rápidas mudanças no cenário tecnológico, a ferramenta deveria poder dar suporte a diferentes linguagens e integrações;
	\item Usabilidade: a ferramenta deveria ser de fácil compreensão e fornecer uma boa experiência ao usuário;
	\item Segurança: é essencial que critérios de segurança sejam inerentes à ferramenta, definição de papéis e usuários.
	
\end{itemize}


\section{Materiais Utilizados}

Tomando por base os critérios mencionados no item 3, passou-se a definição das ferramentas, sendo essas:



\begin{table}[]
	\centering
	\begin{tabular}{|l|c|}
		\hline
		\multicolumn{1}{|c|}{DESCRIÇÃO} & ESPECIFICAÇÃO                                                                                            \\ \hline
		Máquina Física                  & \begin{tabular}[c]{@{}c@{}}Mac OS High Sierra 10.13.4 /\\ i5 2.4Ghz 10 GB RAM 1333 MHz DDR3\end{tabular} \\ \hline
		Oracle VirtualBox               & Versão 5.2.8                                                                                             \\ \hline
		Docker                          & Versão                                                                                                   \\ \hline
		Ansible                         & Versão                                                                                                   \\ \hline
		GitHub                          & Serviço Web: www.github.com                                                                              \\ \hline
	\end{tabular}
\caption{Ferramentas utilizadas}
\end{table}

\begin{itemize}
	
\item VirutalBox: Criação de máquinas virtuais para testes. Serão construídas máquinas virtuais para execução dos hosts clientes e servidores;
\item GitHub: Versionamento e repositório. Esse serviço receberá os códigos das aplicações dos desenvolvedores;
\item Ansible: Gerenciamento de configuração e orquestração da rotina. Essa ferramenta será responsável por administrar as rotinas dentro de todo o processo de deploy. Será executado em uma máquina virtual com Sistema Operacional Ubuntu Linux 64 bits.
\item Jenkins: Automatização da execução de tarefas e testes de implantação. O Jenkins será executado em uma máquina virtual com Sistema Operacional Ubuntu Linux 64 bits.
\item Docker: Contêineres para armazenamento das aplicações e testes. O Docker será executado em uma máquina virtual com Sistema Operacional Ubuntu Linux 64 bits.

\end{itemize}





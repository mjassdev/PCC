\chapter{Justificativa}\label{cap: justificativa}

A visão DevOps traz consigo o ideal de produtividade pela automatização na entrega de um serviço ou aplicação. Aliado a essa proposta, a Integração Contínua (CI) dá suporte para que todo o processo de construção ou build de um código-fonte possa ser realizado de forma automatizada, desde a implantação do ambiente aos testes unitários e de integração, tal prática retira tanto da equipe de desenvolvimento quanto de infra o peso e tempo gasto no ato de disponibilizar essa aplicação para produção.

Fazer uso dessa abordagem possibilita aos times mencionados um retorno rápido acerca das alterações que estão continuamente sendo realizadas no código de um projeto, além de viabilizar uma forma mais barata de resolver problemas relacionados a falhas, quando identificadas.

Falamos então, não somente da mudança do processo em si, mas, mais pontualmente uma mudança cultural. A aplicação de uma visão DevOps traz melhorias frente a compilação do código, testes automatizados, empacotamento, criação de ambientes para teste e produção, configuração da infraestrutura, migração de dados, monitoramento, auditoria, segurança, deploy, entre outros. Algumas empresas que fizeram uso das práticas DevOps passaram a ter feedbacks significativos quanto a adaptação das mudanças do mercado, realizando diversos \textit{deploys} por dia de forma segura.\cite{sato2014devops}

Em se tratando do cenário universitário, é fundamental que a instituição corra na mesma velocidade em que as novas tecnologias surgem, em se tratando de cursos de Tecnologia da Informação isso se torna latente. Novas e boas práticas devem ser experimentadas pelos alunos, a Integração Contínua pode ser aplicada largamente para fins de estudo e experimentos, além de entenderem a vantagem do uso da automatização, podem investir um tempo maior para o desenvolvimento da aplicação.